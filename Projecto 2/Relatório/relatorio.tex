\documentclass{article}
\usepackage[portuges]{babel}
\usepackage[utf8]{inputenc}
\usepackage[margin=1in, includefoot,footskip=30pt,]{geometry} %Equivalente a fullpage sem estragar o cabeçalho e rodapé

\usepackage{graphicx} %Permite inserir imagens
\usepackage{mathtools} % mathtools loads the amsmath package automatically
\usepackage{amssymb}

\DeclarePairedDelimiter\abs{\lvert}{\rvert}%
\DeclarePairedDelimiter\norm{\lVert}{\rVert}%

\usepackage{float}

\usepackage[hidelinks=true]{hyperref}

\usepackage{enumitem} % Permite adicionar tab a itemizes
\usepackage{indentfirst}
\usepackage{subcaption}


% ----- Definiçoes para as tabelas -----
\usepackage{tabularx}
\newcolumntype{Y}{>{\centering\arraybackslash}X}


% ----- Cabeçalho e rodapé -----
\usepackage{color}
%\usepackage{xcolor}
\definecolor{light-gray}{gray}{0.80}
\definecolor{mylight-gray}{gray}{0.30}

\usepackage{fancyhdr}
\pagestyle{fancy}
\fancyhf{}

\renewcommand{\headrulewidth}{1pt}
\renewcommand{\footrulewidth}{0.5pt}

\rhead{\textbf{\large{\textbf{\textsc{Algoritmia e Desempenho em Redes de Computadores}}}} \\ \textsc{Inter-domain routing}}
\lhead{\includegraphics[scale=0.7]{capa/IST_Logo.png}}
\rfoot{\textsc{\textcolor{mylight-gray}{Página} \thepage}}

\usepackage{etoolbox}
\patchcmd{\headrule}{\hrule}{\color{light-gray}\hrule}{}{}
\patchcmd{\footrule}{\hrule}{\color{light-gray}\hrule}{}{}


% ----- Comandos Personalizados -----
\newcommand{\betamaisum}[0]{\ensuremath{\left( \beta + 1 \right)}}

\newcommand{\volt}[0]{\ensuremath{\ V}}
\newcommand{\mvolt}[0]{\ensuremath{\ mV}}
\newcommand{\uvolt}[0]{\ensuremath{\ \mu V}}

\newcommand{\amp}[0]{\ensuremath{\ A}}
\newcommand{\mamp}[0]{\ensuremath{\ mA}}
\newcommand{\uamp}[0]{\ensuremath{\ \mu A}}

\newcommand{\mohm}[0]{\ensuremath{\ m \Omega}}
\newcommand{\ohm}[0]{\ensuremath{\ \Omega}}
\newcommand{\kohm}[0]{\ensuremath{\ k\Omega}}

\newcommand{\siemen}{\ensuremath{\ S}}
\newcommand{\msiemen}{\ensuremath{\ mS}}
\newcommand{\usiemen}{\ensuremath{\ \mu S}}
\newcommand{\Msiemen}{\ensuremath{\ MS}}

\newcommand{\pfarad}{\ensuremath{\ pF}}
\newcommand{\ufarad}{\ensuremath{\ \mu F}}
\newcommand{\nfarad}{\ensuremath{\ nF}}

\newcommand{\Hz}{\ensuremath{\ Hz}}
\newcommand{\kHz}{\ensuremath{\ kHz}}

\newcommand{\dB}{\ensuremath{\ dB}}


\newcommand{\dez}[1]{\ensuremath{\cdot 10^{#1}}}

\newcommand{\barravertical}{\ensuremath{\biggr\rvert}}

% ----- Template para secções subsecções -----
\usepackage[explicit]{titlesec}
\usepackage{color, colortbl}
\usepackage[usenames,dvipsnames]{xcolor}

\titleformat{\section}
	{\normalfont\Large\scshape}{}{0em}{%\setlength
	%\fboxsep{9pt}
	%\colorbox{NavyBlue}
	{\parbox{\dimexpr\textwidth-2\fboxsep\relax}{
	\ifnum\thesection>0
		\thesection\quad#1
	\else
		#1
	\fi}}}

\titleformat{\subsection}{\normalfont\scshape\large}{}{0em}{{#1}}%\thesubsection \ #1}}
\titleformat{\subsubsection}{\normalfont\scshape}{}{0em}{{#1}}%\thesubsubsection \ #1}}
\titleformat{\paragraph}{\normalfont\scshape\small}{}{0em}{{#1}}

%\newcommand{\sectionbreak}{\clearpage} %Faz page break antes de cada secção

\usepackage{pdfpages}
\usepackage[siunitx]{circuitikz}



\usepackage{algpseudocode}
\usepackage{algorithm}

\makeatletter
\renewcommand{\ALG@name}{Algoritmo}
\makeatother

\makeatletter
\newenvironment{breakablealgorithm}
  {% \begin{breakablealgorithm}
   \begin{center}
     \refstepcounter{algorithm}% New algorithm
     \hrule height.8pt depth0pt \kern2pt% \@fs@pre for \@fs@ruled
     \renewcommand{\caption}[2][\relax]{% Make a new \caption
       {\raggedright\textbf{\ALG@name~\thealgorithm} ##2\par}%
       \ifx\relax##1\relax % #1 is \relax
         \addcontentsline{loa}{algorithm}{\protect\numberline{\thealgorithm}##2}%
       \else % #1 is not \relax
         \addcontentsline{loa}{algorithm}{\protect\numberline{\thealgorithm}##1}%
       \fi
       \kern2pt\hrule\kern2pt
     }
  }{% \end{breakablealgorithm}
     \kern2pt\hrule\relax% \@fs@post for \@fs@ruled
   \end{center}
  }
\makeatother


% ----- DOCUMENTO -----
\begin{document}

% Inclui a capa (existem pacotes especificos que interferem com os existentes neste documento)
\includepdf[pages={1}]{capa/capa.pdf}
\setcounter{page}{1}

\subsection{Algoritmo \texttt{CommercialRoute}}

Com este algoritmo pretende-se descobrir qual o tipo de rota comercial mais favorável a um determinado domínio para chegar a um dado domínio de destino.

\begin{algorithm}[H]
\caption{}
\label{al:generatecode}
\begin{algorithmic}
\Function{CommercialRoute}{$s$}
	\For {all vertices $v$}
		\State $visisted[v] := false$
	\EndFor
	
	\State $visisted[s] := true$
	\State $domain := s$; $domain\_route := Itself$
	
	\State $fifo\_clients := \emptyset$; 
	$fifo\_peers := \emptyset$;
	$fifo\_providers := \emptyset$
	
	\State
	
	\While{$domain \neq NIL $}		

		\State\Call{EvaluateLinks}{all vertices $v$ clients of $domain$, $fifo\_providers$, $visited$}
				
		\If {$domain\_route = Client$ \textbf{or} $domain\_route := Itself$}
			\State\Call{EvaluateLinks}{all vertices $v$ peers of $domain$, $fifo\_peers$, $visited$}			
			\State\Call{EvaluateLinks}{all vertices $v$ providers of $domain$, $fifo\_clients$, $visited$}
		\EndIf
		
		\State\Call{Output}{$domain$, $domain\_route$}
		\State
		\State $domain := \Call{RemoveFifo}{fifo\_clients}$
		\State $domain\_route := Client$
		\If { $domain = NIL$ }
			\State $domain := \Call{RemoveFifo}{fifo\_peers}$
			\State $domain\_route := Peer$
			\If { $domain = NIL$ }
				\State $domain := \Call{RemoveFifo}{fifo\_providers}$
				\State $domain\_route := Provider$
			\EndIf
		\EndIf		
	\EndWhile
\EndFunction
\State
\Function {EvaluateLinks}{$vertices$, $destination\_list$, $visited$}
	\For {all vertices $v$}
		\If {$visited[v] = false$}
			\State $visited[v] := true$
			\State $destination\_list := destination\_list \cup v$
		\EndIf			
	\EndFor
\EndFunction
\end{algorithmic}
\end{algorithm}

A representação dos domínios e as suas ligações não constitui uma árvore devido às ligações entre pares, que permitem a existência de ciclos violando assim a condição de árvore. No entanto, não é possível é possível que um domínio seja fornecedor de um domínio do qual um dos seus fornecedores depende




O funcionamento deste algoritmo é baseado no algoritmo de pesquisa em largura em árvores: BFS (\textit{Breadth-first search}).



\end{document}

