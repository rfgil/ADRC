\documentclass{article}
\usepackage[portuges]{babel}
\usepackage[utf8]{inputenc}
\usepackage[margin=1in, includefoot,footskip=30pt,]{geometry} %Equivalente a fullpage sem estragar o cabeçalho e rodapé

\usepackage{graphicx} %Permite inserir imagens
\usepackage{mathtools} % mathtools loads the amsmath package automatically
\usepackage{amssymb}

\DeclarePairedDelimiter\abs{\lvert}{\rvert}%
\DeclarePairedDelimiter\norm{\lVert}{\rVert}%

\usepackage{float}

\usepackage[hidelinks=true]{hyperref}

\usepackage{enumitem} % Permite adicionar tab a itemizes
\usepackage{indentfirst}
\usepackage{subcaption}


% ----- Definiçoes para as tabelas -----
\usepackage{tabularx}
\newcolumntype{Y}{>{\centering\arraybackslash}X}


% ----- Cabeçalho e rodapé -----
\usepackage{color}
%\usepackage{xcolor}
\definecolor{light-gray}{gray}{0.80}
\definecolor{mylight-gray}{gray}{0.30}

\usepackage{fancyhdr}
\pagestyle{fancy}
\fancyhf{}

\renewcommand{\headrulewidth}{1pt}
\renewcommand{\footrulewidth}{0.5pt}

\rhead{\textbf{\large{\textbf{\textsc{Algoritmia e Desempenho em Redes de Computadores}}}} \\ \textsc{Edge-connectivity in graphs}}
\lhead{\includegraphics[scale=0.7]{capa/IST_Logo.png}}
\rfoot{\textsc{\textcolor{mylight-gray}{Página} \thepage}}

\usepackage{etoolbox}
\patchcmd{\headrule}{\hrule}{\color{light-gray}\hrule}{}{}
\patchcmd{\footrule}{\hrule}{\color{light-gray}\hrule}{}{}


% ----- Comandos Personalizados -----
\newcommand{\betamaisum}[0]{\ensuremath{\left( \beta + 1 \right)}}

\newcommand{\volt}[0]{\ensuremath{\ V}}
\newcommand{\mvolt}[0]{\ensuremath{\ mV}}
\newcommand{\uvolt}[0]{\ensuremath{\ \mu V}}

\newcommand{\amp}[0]{\ensuremath{\ A}}
\newcommand{\mamp}[0]{\ensuremath{\ mA}}
\newcommand{\uamp}[0]{\ensuremath{\ \mu A}}

\newcommand{\mohm}[0]{\ensuremath{\ m \Omega}}
\newcommand{\ohm}[0]{\ensuremath{\ \Omega}}
\newcommand{\kohm}[0]{\ensuremath{\ k\Omega}}

\newcommand{\siemen}{\ensuremath{\ S}}
\newcommand{\msiemen}{\ensuremath{\ mS}}
\newcommand{\usiemen}{\ensuremath{\ \mu S}}
\newcommand{\Msiemen}{\ensuremath{\ MS}}

\newcommand{\pfarad}{\ensuremath{\ pF}}
\newcommand{\ufarad}{\ensuremath{\ \mu F}}
\newcommand{\nfarad}{\ensuremath{\ nF}}

\newcommand{\Hz}{\ensuremath{\ Hz}}
\newcommand{\kHz}{\ensuremath{\ kHz}}

\newcommand{\dB}{\ensuremath{\ dB}}


\newcommand{\dez}[1]{\ensuremath{\cdot 10^{#1}}}

\newcommand{\barravertical}{\ensuremath{\biggr\rvert}}

% ----- Template para secções subsecções -----
\usepackage[explicit]{titlesec}
\usepackage{color, colortbl}
\usepackage[usenames,dvipsnames]{xcolor}

\titleformat{\section}
	{\normalfont\Large\scshape}{}{0em}{%\setlength
	%\fboxsep{9pt}
	%\colorbox{NavyBlue}
	{\parbox{\dimexpr\textwidth-2\fboxsep\relax}{
	\ifnum\thesection>0
		\thesection\quad#1
	\else
		#1
	\fi}}}

\titleformat{\subsection}{\normalfont\scshape\large}{}{0em}{{#1}}%\thesubsection \ #1}}
\titleformat{\subsubsection}{\normalfont\scshape}{}{0em}{{#1}}%\thesubsubsection \ #1}}
\titleformat{\paragraph}{\normalfont\scshape\small}{}{0em}{{#1}}

%\newcommand{\sectionbreak}{\clearpage} %Faz page break antes de cada secção

\usepackage{pdfpages}

\usepackage{xcolor}
\usepackage{pgfplots}
\usepackage{tikz}

\usepackage{bchart}

% Define bar chart colors
%
\definecolor{bblue}{HTML}{4F81BD}
\definecolor{rred}{HTML}{C0504D}
\definecolor{ggreen}{HTML}{9BBB59}
\definecolor{ppurple}{HTML}{9F4C7C}



\usepackage{algpseudocode}
\usepackage{algorithm}

\makeatletter
\renewcommand{\ALG@name}{Algoritmo}
\makeatother

\makeatletter
\newenvironment{breakablealgorithm}
  {% \begin{breakablealgorithm}
   \begin{center}
     \refstepcounter{algorithm}% New algorithm
     \hrule height.8pt depth0pt \kern2pt% \@fs@pre for \@fs@ruled
     \renewcommand{\caption}[2][\relax]{% Make a new \caption
       {\raggedright\textbf{\ALG@name~\thealgorithm} ##2\par}%
       \ifx\relax##1\relax % #1 is \relax
         \addcontentsline{loa}{algorithm}{\protect\numberline{\thealgorithm}##2}%
       \else % #1 is not \relax
         \addcontentsline{loa}{algorithm}{\protect\numberline{\thealgorithm}##1}%
       \fi
       \kern2pt\hrule\kern2pt
     }
  }{% \end{breakablealgorithm}
     \kern2pt\hrule\relax% \@fs@post for \@fs@ruled
   \end{center}
  }
\makeatother


% ----- DOCUMENTO -----
\begin{document}

% Inclui a capa (existem pacotes específicos que interferem com os existentes neste documento)
\includepdf[pages={1}]{capa/capa.pdf}
\setcounter{page}{1}

\section{Implementação}

Pretende-se implementar um algoritmo que permita determinar o conjunto mínimo de arestas de um grafo não direcionado de forma a separar dois dados nós. Por outras palavras, pretende-se encontrar o corte mínimo que coloque os dois nós em conjuntos distintos.

Sabe-se da teoria que calcular um corte mínimo é possível através do calculo do fluxo máximo de um grafo que pode ser computado recorrendo ao algoritmo de \textit{Ford–Fulkerson}.

Este algoritmo permite calcular o fluxo máximo entre dois nós num grafo direcionado. Para o adaptar a um grafo não direcionado é necessário considerar que cada aresta corresponde a duas ligações em sentidos opostos. Uma vez que todas as arestas são igualmente utilizáveis define-se um fluxo máximo de 1 para todas elas e com isto o calculco de fluxo máximo corresponderá também ao número de caminhos dijsuntos desde o nó de origem até ao destino.

Com estas definições é então possível implementar o algoritmo de \textit{Ford–Fulkerson}.

\begin{algorithm}[H]
\caption{Ford–Fulkerson}
\label{al:Fordfulkerson}
\begin{algorithmic}[1]
\Function{getDisjointPaths}{$G$}
	\For {all edges $uv$}
		\State $f[u,v] := 0$
		\State $f[v,u] := 0$
	\EndFor
	
	\While {There is a path $P$ from $s$ to $t$ in the residual network $G_{f}$}
	\Comment Augmenting path
    \For {each $uv \in E(P)$}
      \State $f[u,v] := 1$
	  	\State $f[v,u] := 0$
    \EndFor
	\EndWhile
	
	\State \textbf{return} \Call{Count}{flux entering node $t$}
\EndFunction
\end{algorithmic}
\end{algorithm}

Este algoritmo começa por inicializar os valores de fluxo em todas as arestas. De seguida inicia a sua busca por caminhos aumentativos. Sempre que é encontrado um caminho desde a origem até ao destino, tendo em conta o fluxo máximo que pode atravessar cada aresta, o fluxo das arestas que o constituem são colocados ao máximo: 1.

\begin{algorithm}[H]
\caption{Augmenting Path}
\label{al:AugmentingPath}
\begin{algorithmic}[1]
\Function{Augmenting\_Path}{$G$, $visited$, $u$}
  
  \State $visited[u] =$ \textbf{true}
  
  \If {$u = s$}
    \State \textbf{return true}
  \EndIf
  
  \For {all edges, $uv$, of $u$}
    \If {\textbf{not} $visited[v]$ \textbf{and} \Call{Augmenting\_Path}{$G$, $visited$, $v$}}
      \State $f[u,v] := 1$
	  	\State $f[v,u] := 0$
	  	\State \textbf{return true}
    \EndIf
  \EndFor  
	
	\State \textbf{return} \textbf{false}
\EndFunction
\end{algorithmic}
\end{algorithm}

A função para encontrar caminhos aumentativos, simplificada nas linhas 6 a 10 do algoritmo \ref{al:Fordfulkerson}, baseia-se no algoritmo de procura em profundidade, BFS. E é implementada por recorrencia, como se observa com maior detalhe no algoritmo \ref{al:AugmentingPath}.

Uma vez que as capacidades de todas as ligações é um numero inteiro é garantido que o algoritmo de \textit{Ford–Fulkerson} termine. A complexidade deste algoritmo é $\mathcal{O}(n\cdot m^2)$ uma vez que é necessário procurar no limite $n\cdot m$ caminhos aumentativos e para cada caminho aumentativo são necessárias $m$ iterações. 

Com isto é possível calcular o número de caminhos disjuntos desde a origem até ao destino e obter o grafo residual. O grafo residual é importante para a próxima etapa em que se pretende descobrir qual o conjunto mínimo de arestas a remover para separar estes dois nós.

Isto porque depois de calculado o fluxo máximo é garantido que não existe um caminho no grafo residual desde a origem até ao destino. Assim sendo para descobrir as arestas do corte mínimo é necessário descobrir quais permitem a ligação entre nós a que é possível chegar (no grafo residual) a partir da origem e os nós a que é possível chegar a partir do destino.

O algoritmo para conseguir este resultado é uma simples BFS. No entanto os nós em vez de marcados como visitados são marcados com um número que identifica se lhes foi possivel chegar a partir da origem ou do destino.

\begin{algorithm}[H]
\caption{}
\label{al:walkResidualGraph}
\begin{algorithmic}[1]
\Function{walkResidualGraph}{$G$, $visited$, $u$}
  
  \State $visited[u] = ORIGIN$
   
  \For {all edges, $uv$, of $u$}
    \If {$visited[v]$ != $ORIGIN$ \textbf{and} \Call{canCrossResidualGraph}{$uv$}}
      \State \Call{walkResidualGraph}{$G$, $visited$, $v$}
    \EndIf
  \EndFor  
\EndFunction
\end{algorithmic}
\end{algorithm}

Utilizando o mesmo vetor $visited$, é agora executado o seguinte algoritmo \ref{al:findMinimumCut}. Desta vez a partir do nó de destino.


\begin{algorithm}[H]
\caption{}
\label{al:findMinimumCut}
\begin{algorithmic}[1]
\Function{findMinimumCut}{$G$, $visited$, $u$}
  
  \State $visited[u] = DESTINATION$
   
  \For {all edges, $uv$, of $u$}
    \If {$visited[v]$ = $ORIGIN$}
      \State $uv$ belongs to the minimum cut set \\
\ \ \ \ \ \ \ \ \  \textbf{else if} {$visited[v]$ != $DESTINATION$ \textbf{and} \Call{canCrossResidualGraph}{$uv$}} \textbf{then}
      \State \Call{findMinimumCut}{$G$, $visited$, $v$}
    \EndIf
  \EndFor  
\EndFunction
\end{algorithmic}
\end{algorithm}

Este conjunto de algoritmos tem uma complexidade de $\mathcal{O}(m)$ uma vez que são visitados todos os nós e todas as suas arestas.

De seguida pretende-se uma nova funcionalidade para o programa. Pretende-se que seja possível calcular a quantidade de pares de nós cuja separação tenha tenha $k$ arestas. Para calcular este resultado é apenas necessário executar o algoritmo \ref{al:Fordfulkerson}, \texttt{getDisjointPaths}, para todas as combinações de pares de nós.


\begin{figure}[H]
\centering
\begin{subfigure}{.3\textwidth}
  \centering
  \includegraphics[width=.9\linewidth]{grafo.png}
  \caption{Grafo}
  \label{fig:sub1}
\end{subfigure}%
\begin{subfigure}{.6\textwidth}
  \centering
  \begin{bchart}[step=20,max=100, unit=\%]
        \bclabel{$k$}
        \bcbar[text=1]{34.55}%19 {\footnotesize $34,55\%$ }}
            \smallskip
        \bcbar[text=2]{54.55} %{\footnotesize $54,55\%$ }}
            \smallskip
        \bcbar[text=3]{10.90} %{\footnotesize $10,90\%$ }}
    \end{bchart}
  \caption{Percentagem de pares de nós separados por $k$ arestas.}
  \label{fig:sub2}
\end{subfigure}
\caption{Exemplo}
\label{fig:test}
\end{figure}

Por último é pedido que seja implementada uma função que calcule a \textit{edge-connectivity} do grafo em estudo e forneça um conjunto de arestas que sustente essa afirmação. A \textit{edge-connectivity} é o número mínimo de arestas que é necessário remover para separar o grafo em dois conjuntos diferentes. Uma vez mais é necessário executar o algoritmo \ref{al:Fordfulkerson}, \texttt{getDisjointPaths}, para todas as combinações de pares de nós. 

Deve ser inicializada uma variável chamada \texttt{edge-connectivity} com o valor infinito. Sempre que for encontrado um par de nós com um número de caminhos disjuntos inferior a \texttt{edge-connectivity}, o valor desta é atualizado para o número de caminhos disjuntos obtido e de seguida deve ser calculado qual o corte mínimo associado.

\section{Discussão}

O tópico dos fluxos é importante na medida em que existem vários problemas que podem ser reduzidos a esta topologia e assim resolvidos tendo por base as suas definições. Com este trabalho esteve-se diretamente em contacto com este facto, uma vez que o problema do conjunto de arestas que forma o corte mínimo de um grafo não está à primeira vista relacionado com um problema de fluxo.

Foi também possível tomar consciência do custo computacional associado a este algoritmo de complexidade polinomial. Uma vez problemas com uma quantidade considerável de vértices implica algum tempo de espera pelo resultado do algoritmo.

\end{document}
